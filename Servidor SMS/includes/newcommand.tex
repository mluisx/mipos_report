\newcommand{\si}[1]{\raisebox{0ex}{\scriptsize #1}}
\newcommand{\sm}[1]{$_{#1}$}
\newcommand{\sit}[1]{\raisebox{0ex}{\tiny #1}}
\newcommand{\Si}[1]{\raisebox{+0.6ex}{\scriptsize #1}}
%\newcommand{\bneg}[1]{$\overline{\textsf{#1}}$}

%\newcommand{\bp}[1]{\footnotesize{\textnormal{\textbf{#1}}}\normalsize}
\newcommand{\bp}[1]{{\footnotesize \textnormal{\textbf{#1}}}} % Escribe mas chiquito
%\newcommand{\bn}[1]{\footnotesize{$\overline{\textnormal{\textbf{#1}}}$}\normalsize}
\newcommand{\bn}[1]{{\footnotesize $\overline{\textnormal{\textbf{#1}}}$}}
%\newcommand{\bpn}[2]{\footnotesize{\textbf{#1}/$\overline{\textnormal{\textbf{#2}}}$}\normalsize} 
\newcommand{\bpn}[2]{{\footnotesize \textbf{#1}/$\overline{\textnormal{\textbf{#2}}}$}} 
%\newcommand{\reg}[1]{\small{\textnormal{\textbf{#1}}}\normalsize}
\newcommand{\reg}[1]{{\small \textnormal{\textbf{#1}}}}
\newcommand{\mrx}[1]{\scriptsize{#1}}
%\newcommand{\rb}[2]{#2\,(reg. \reg{#1})}

\newcommand{\bpt}[1]{{\scriptsize\hspace{1mm}\textnormal{\textbf{#1}}}\hspace{1mm}}
\newcommand{\bnt}[1]{{\scriptsize\hspace{1mm}$\overline{\textnormal{\textbf{#1}}}$\hspace{1mm}}}


\newcommand{\bpf}[1]{\textnormal{\textbf{\footnotesize{#1}}}}
\newcommand{\bnf}[1]{\overline{\textnormal{\textbf{\footnotesize{#1}}}}}
\newcommand{\bpnf}[2]{\textnormal{\textbf{\footnotesize{#1/}}}\overline{\textnormal{\textbf{\footnotesize{#2}}}}} 
\newcommand{\regf}[1]{\textnormal{\textbf{\small{#1}}}}


\newcommand{\bnbf}[1]{\bn{#1}}
\newcommand{\bpnbf}[2]{\bpn{#1}{#2}} 

%\newcommand{\bnbf}[1]{$\overline{\textbf{#1}}$}
%\newcommand{\bpnbf}[2]{\textbf{#1/}$\overline{\textbf{#2}}$} 

%\newcommand{\fb}[1]{\


\newcommand{\tablaochobits}[9]{
	\begin{tabular}{cccccccc}
		#1 \\
		\hline
		\multicolumn{1}{|c}{#2} & \multicolumn{1}{|c}{#3} &  \multicolumn{1}{|c}{#4} & \multicolumn{1}{|c}{#5} &  \multicolumn{1}{|c}{#6} &  \multicolumn{1}{|c}{#7} & \multicolumn{1}{|c}{#8} & \multicolumn{1}{|c|}{#9}  \\ 
		\hline
		bit7 & & & & & & & bit0 \\
	\end{tabular}
}	
\newcommand{\tablaochobitsin}[8]{
	\begin{tabular}{cccccccc}
		\hline
		\multicolumn{1}{|c}{#1} & \multicolumn{1}{|c}{#2} &  \multicolumn{1}{|c}{#3} & \multicolumn{1}{|c}{#4} &  \multicolumn{1}{|c}{#5} &  \multicolumn{1}{|c}{#6} & \multicolumn{1}{|c}{#7} & \multicolumn{1}{|c|}{#8}  \\ 
		\hline
		bit7 & & & & & & & bit0 \\
	\end{tabular}
}	

\newcommand{\acfn}[1]{\acsu{#1}\footnote{\acl{#1}}}
\newcommand{\aci}[1]{\acsu{#1} (\acl{#1})}

%\newcommand{\ud}[1]{\,\textnormal{\small{{#1}}}}
\newcommand{\un}[1]{\,\textnormal{\small{{#1}}}}
\newcommand{\unscript}[1]{\,\textnormal{\scriptsize{{#1}}}}

\newenvironment{tabla}[2]%
{%
\begin{table}[htbp]
	\rule{\textwidth}{.4mm}%
	\caption{#1}%
	\label{#2}%
	\centering%
	\vspace{4mm}%
}%
{%
	\rule{\textwidth}{.4mm}%
\end{table}
}%

\newenvironment{tablao}[3]%
{%
\begin{table}[#3]
	\rule{\textwidth}{.4mm}%
	\caption{#1}%
	\label{#2}%
	\centering%
	\vspace{4mm}%
}%
{%
	\rule{\textwidth}{.4mm}%
\end{table}
}%

\newenvironment{tablaot}[3]% Tabla Sin La Linea De Arriba
{%
\begin{table}[#3]
	\caption{#1}%
	\label{#2}%
	\centering%
	\vspace{4mm}%
}%
{%
	\rule{\textwidth}{.4mm}%
\end{table}
}%

\newenvironment{tablaob}[3]% Tabla Sin La Linea De Abajo
{%
\begin{table}[#3]
	\rule{\textwidth}{.4mm}%
	\caption{#1}%
	\label{#2}%
	\centering%
	\vspace{4mm}%
}%
{%
\end{table}
}%

\newenvironment{tablaos}[3]% Tabla Con Lineas Gruesas
{%
\begin{table}[#3]
	\rule{\textwidth}{1mm}%
	\caption{#1}%
	\label{#2}%
	\centering%
	\vspace{4mm}%
}%
{%
	\rule{\textwidth}{1mm}%
\end{table}
}%

\newenvironment{tablaol}[3]% Tabla Sin Lineas
{%
\begin{table}[#3]
	\caption{#1}%
	\label{#2}%
	\centering%
	\vspace{4mm}%
}%
{%
\end{table}
}%

\newcommand{\figura}[3]{
\begin{figure}[htbp]%
	\rule{\textwidth}{.4mm}\\[5mm]%
	\centering%
	\includegraphics{#3}%
	\caption{#1}%
	\label{#2}%
	\rule{\textwidth}{.4mm}%
\end{figure}%
}

\newcommand{\figuraop}[4]{
\begin{figure}[htb]%
	\rule{\textwidth}{.4mm}\\[5mm]%
	\centering%
	\includegraphics[#3]{#4}%
	\caption{#1}%
	\label{#2}%
	\rule{\textwidth}{.4mm}%
\end{figure}%
}

\newcommand{\figuraoo}[5]{ %figura con dos lineas
\begin{figure}[#5]%
	\rule{\textwidth}{.4mm}\\[5mm]%
	\centering%
	\includegraphics[#3]{#4}%
	\caption{#1}%
	\label{#2}%
	\rule{\textwidth}{.4mm}%
\end{figure}%
}


\newcommand{\figuraot}[5]{ % Figura Sin La Linea De Arriba
\begin{figure}[#5]%
	\centering%
	\includegraphics[#3]{#4}%
	\caption{#1}%
	\label{#2}%
	\rule{\textwidth}{.4mm}%
\end{figure}%
}

\newcommand{\figuraob}[5]{ % Figura Sin La Linea De Abajo
\begin{figure}[#5]%
	\rule{\textwidth}{.4mm}\\[5mm]%
	\centering%
	\includegraphics[#3]{#4}%
	\caption{#1}%
	\label{#2}%
\end{figure}%
}

\newcommand{\figuraos}[5]{ % Figura Sin Lineas
\begin{figure}[#5]%
	\centering%
	\includegraphics[#3]{#4}%
	\caption{#1}%
	\label{#2}%
\end{figure}%
}

\newenvironment{mylisting}
{\begin{list}{}{\setlength{\leftmargin}{1em}}\item\scriptsize\bfseries}
{\end{list}}

\newenvironment{mytinylisting}
{\begin{list}{}{\setlength{\leftmargin}{1em}}\item\tiny\bfseries}
{\end{list}}

\newenvironment{codigofuente}[2]%
{%
\begin{tabla}{#1}{#2}%
%	\begin{mylisting}%
	\begin{flushleft}
		\ttfamily%
		\mdseries%
		\scriptsize%
\begin{sloppypar}
}%
{%
\end{sloppypar}
		\normalsize
		\normalfont
	\end{flushleft}
%	\end{mylisting}%
\end{tabla}%
}%

\definecolor{color-primitiva}{rgb}{.5,.1,.3}
\definecolor{color-clase}{rgb}{0,0,0}
\definecolor{color-numero}{rgb}{0,.5,0}
\definecolor{color-string}{rgb}{.9,.8,0}
\definecolor{color-preproc}{rgb}{.8,.9,0}

%\newcommand{\cfpr}[1]{\textcolor{color-primitiva}{\texttt{\textbf{#1}}}}
%\newcommand{\cfcl}[1]{\textcolor{color-clase}{\texttt{\textbf{#1}}}}
%\newcommand{\cfnu}[1]{\textcolor{color-numero}{\texttt{\textbf{#1}}}}
%\newcommand{\cfst}[1]{\textcolor{color-string}{\texttt{#1}}}
%\newcommand{\cfpp}[1]{\textcolor{color-preproc}{\texttt{#1}}}


\newcommand{\cfa}[1]{\guillemotleft\texttt{#1}\guillemotright} % C�digo Fuente: Archivo
\newcommand{\cfc}[1]{\guilsinglleft\texttt{#1}\guilsinglright} % C�digo fuente: C�digo
\newcommand{\cfcl}[1]{\texttt{#1}}
\newcommand{\cf}[1]{\texttt{#1}}
\newcommand{\cfm}[1]{\texttt{\textbf{#1}}}


\newcommand{\etclass}[1]{\cfcl{#1}} % Estilo de texto para los nombres de las clases en C++
\newcommand{\cfobj}[1]{\textsl{\texttt{#1}}} % Estilo de texto para los nombres de los objetos en C++
%\newcommand{\etclass}[1]{\textbf{#1}} % Estilo de texto para los nombres de las clases en C++
%\newcommand{\etobj}[1]{\textsl{#1}} % Estilo de texto para los nombres de los objetos en C++
