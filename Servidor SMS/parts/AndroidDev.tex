\chapter{El desarrollo en Android}
\label{cpt:android.dev}

En este cap\'itulo se describen las distintas partes que componen una aplicaci\'on Android y como la plataforma se encuentra estructurada para poder dar soporte y manejo a las mismas.

\section{La \ac{DVM}}
\label{sec:dvm}

Dalvik es la m\'aquina virtual que utiliza la plataforma para dispositivos m\'oviles Android. El dise\~no fue realizado por Dan Bornstein con contribuciones de otros ingenieros de Google.

Dalvik est\'a optimizada para requerir poca memoria y est\'a diseñada para permitir la ejecuci\'on de varias instancias de la m\'aquina virtual simult\'aneamente, delegando en el sistema operativo subyacente el soporte de aislamiento de procesos, gesti\'on de memoria e hilos.

A menudo, Dalvik es nombrada como una m\'aquina virtual Java, pero esto no es estrictamente correcto, ya que el bytecode con el que opera no es Java bytecode. Sin embargo, la herramienta dx incluida en el \ac{SDK} de Android permite transformar los archivos \texttt{.class} de Java compilados por un compilador Java al formato de archivos \texttt{.dex}.

El nombre de Dalvik fue elegido por Bornstein en honor a Dalvík, un pueblo de Islandia donde vivieron antepasados suyos.~\cite{wiki.dvm}

\section{La arquitectura Android}
\label{sec:arq}

La arquitectura del sistema operativo esta compuesto por las siguientes caracter\'isticas (ver figura \ref{fig:android.architecture}):

\figuraot{La arquitectura Android}{fig:android.architecture}{width=\textwidth}{Android-system-architecture}{!h}

\begin{itemize}
\item Aplicaciones (escritas en Java, ejecutadas en Dalvik).
\item Librer\'ias y servicios del framework (escritas en su mayoria en Java).
\subitem Las aplicaciones y la mayor parte del c\'odigo del framework ejecutadas en la m\'aquina virtual. 
\item Librerias nativas, daemons y servicios (escritos en C o C++).
\item El kernel de linux, que incluye drivers para:
\subitem hardware
\subitem redes
\subitem sistema de acceso a archivos
\subitem \ac{IPC}
\end{itemize}

\section{Partes que componen un proyecto de desarrollo Android}
\label{sec:dev.components}

En esta secci\'on se presentan las distintas clases, vistas y recursos. Como interactuan cada uno de ellos y cual es el prop\'osito de cada uno.

\subsection{Las activities}
\label{subsec:dev.activities}

Una activity conforma las siguientes caracter\'isticas:

\begin{itemize}
\item Consiste en una pantalla dentro de una aplicaci\'on.
\item Una aplicaci\'on puede contener una o m\'as Activities.
\item Pueden ser invocadas desde otras aplicaciones.
\item Cada activity es independiente de las dem\'as.
\item Todas las Activities extienden de la clase Activity.
\item Cada Activity tiene una ventana en la cual mostrarse.
\item Normalmente ocupa toda la pantalla.
\item Puede utilizar ventanas adicionales como dialogs.
\item Se muestra utilizando un \'arbol de vistas que extienden de la clase view.
\item El ra\'iz del \'arbol se define llamando al m\'etodo \texttt{setContentView}.
\end{itemize}

\subsection{El ciclo de vida de una activity}
\label{subsec:dev.activity.lifecycle}

Una activity tiene tres estados principales:

\begin{enumerate}
\item Active, actualmente en pantalla con foco.
\item Paused, perdi\'o foco, contin\'ua parcialmente visible.
\item Stopped, fue tapado por otra Activity.
\end{enumerate}

Las transiciones de estado son notificadas al activity por una serie de m\'etodos protegidos. Se deber\'a mantener una llamada al m\'etodo original en la clase padre al sobreescribir cualquiera de estos m\'etodos.

Los siete m\'etodos del ciclo de vida (figura \ref{fig:android.activity.lifecycle}) pueden ser utilizados para monitorear los siguientes subciclos.

\begin{itemize}
\item Tiempo de vida, usado para liberar recursos en el \texttt{onDestroy} que fueron creados en el \texttt{onCreate}.
\item Visibilidad, para mantener recursos que afecten la UI.
\item Primer plano, para mantener la interactividad con el usuario.
\end{itemize}

\figuraot{Ciclo de vida de una activity}{fig:android.activity.lifecycle}{width=\textwidth}{activity_lifecycle}{!h}

\subsection{El back stack de activities}
\label{subsec:dev.activity.stack}

Las activities son administradas mediante una pila denominada back stack en el orden en el cual cada una de estas activities se va creando.

Cuando una activity en pantalla inicia otra, la nueva activity se coloca al tope del back stack y toma el foco de la pantalla. La actividad anterior se mantiene en el back stack, pero se detiene. Cuando se ingresa al estado de stop, el sistema retiene el estado actual de su \ac{UI}.

En el momento en el que el usuario presiona el bot\'on de retroceso del dispositivo, la actividad actual se destruye y la anterior a esta ejecuta el m\'etodo \texttt{resume} para volver recuperar su estado y pueda ser visualizado en pantalla.

De ninguna forma las activities pueden cambiar su posici\'on dentro del back stack, ya que justamente se opera como una pila de \'ultimo en entrar, primero en salir. La figura \ref{fig:android.activity.lifecycle} visualiza el comportamiento en el tiempo mostrando el progreso entre las distintas activities y como el back stack se encuentra formado en las distintas etapas.

\figuraoo{Diagrama del back stack de activities}{fig:android.stackActivities}{width=\textwidth}{diagram_backstack}{!h}

\subsection{Los Intents}
\label{subsec:dev.intents}

Un intent es una descripci\'on abstracta de una operaci\'on a ser ejecutada. Puede ser utilizado con el me\'todo \texttt{startActivity} para transferir informaci\'on entre distintas activities, como una de las funciones m\'as importantes, entre otras m\'as.

Por medio de un intent, se hace disponible realizar late binding en tiempo de ejecuci\'on entre el c\'odigo en diferentes aplicaciones. Basicamente consiste en una estructura de datos pasiva conteniendo una descripci\'on abstracta de una acci\'on a ser realizada.

Un intent puede referenciar un componente, y si la referencia no es expl\'icita Android buscar\'a el componente que mejor responda.

\subsection{El Manifest}
\label{subsec:dev.manifest}

El manifest consiste en un archivo en formato \texttt{.xml} donde cada aplicaci\'on describe y declara todos sus componentes, sus permisos que requerira para acceder a partes protegidas de la \ac{API} y su interacci\'on con otras aplicaciones, su nivel m\'inimo de \ac{API} Android requerido y todas las librer\'ias a las que la aplicaci\'on hace uso.

Cada aplicaci\'on tiene su manifest que se incluir\'a en el archivo \texttt{.apk}.

\subsection{Los layouts}
\label{subsec:dev.layouts}

El layout define la distribuci\'on de las vistas y contenedores, como la \ac{UI} de una activity on una aplicaci\'on tipo widget.

Los layouts pueden definirse:

\begin{itemize}
\item en el formato \texttt{.xml}: brindando un vocabulario que corresponde a las clases y subclases de View, aquellos los cuales est\'an definidos para layouts y widgets.
\item instanciando los elementos en tiempo de ejecuci\'on: la aplicaci\'on puede crear objetos del tipo View y ViewGroup (y manipular sus propiedades) programaticamente.
\end{itemize}

Cada tipo de layout se encuentra ubicado en una jerarqu\'ia de vistas (views) (figura \ref{fig:android.layouts}) con par\'ametros de layout asociados a cada una de estas vistas. 

\figuraoo{Jerarqu\'ia de vistas y layouts}{fig:android.layouts}{width=\textwidth}{layoutparams}{!h}

\subsection{Los eventos de \ac{UI}}
\label{subsec:dev.ui.events}

Son interfaces en la clase \texttt{View} para registrar distintos listeners para los siguientes eventos:

\begin{itemize}
\item \texttt{onClick}: Cuando el usuario toca el item o hace foco con las teclas de navegaci\'on de un dipositivo con teclado y presiona la tecla enter.
\item \texttt{onLongClick}: Cuando el usuario toca el item y lo mantiene presionado por un tiempo determinado, o tambi\'en al mantener presionado la tecla enter. 
\item \texttt{onFocusChange}: Cuando el usuario navega y toma el foco del item o lo pierde.
\item \texttt{onKey}: Cuando el usuario est\'a haciendo foco en un item y presiona o suelta una tecla o bot\'on en el dispositivo.
\item \texttt{onTouch}: Cuando el usuario realiza una acci\'on denominada touch event, como por ejemplo, un movimiento espec\'ifico al tocar la pantalla (movement gesture).
\item \texttt{onCreateContextMenu}: Cuando un men\'u de contexto es creado.
\end{itemize}

\subsection{El style}
\label{subsec:dev.style}

Es una colecci\'on de propiedades que especifican el formato de una vista. Un style puede aplicarse a un \texttt{View} individual o a una \texttt{Activity} completa. Estos se definen en formato \texttt{xml}.

\subsection{El theme}
\label{subsec:dev.theme}

Cuando un style es aplicado a un \texttt{Activity} o \texttt{Aplication} y no a un \texttt{View} se lo denomina theme.

Todas las \texttt{Views} del \texttt{Activity} o \texttt{Application} tendr\'an el Style aplicado. Como ejemplo, al aplicar un estilo de fuente como un theme para una \texttt{Activity}, todos los textos que se encuentren dentro de esa \texttt{Activity} tendr\'an aplicado el mismo tipo de fuente. 

\subsection{El AdapterView}
\label{subsec:dev.adapterView}

Es una subclase de \texttt{ViewGroup} cuyas \texttt{Views} hijas son definidas por un Adapter. Su funci\'on es la de poder mostrar datos almacenados.

Algunos \texttt{AdapterView} son:

\begin{itemize}
\item Gallery: Una vista que muestra items en una lista que puede realizar scroll y que se encuentran centralizados horizontalmente.
\item ListView: Una vista que muestra items en una lista vertical de tipo scroll.
\item Spinner: Una vista que muestra un hijo a la vez y permite al usuario elegir uno de estos como cualquier lista.
\end{itemize}

El \texttt{Adapter} es una clase utilizada por un \texttt{AdapterView} para obtener datos y construir las vistas hijas.

\subsection{Los men\'ues}
\label{subsec:dev.menus}

Los men\'ues son un tipo de \ac{UI} muy com\'un utilizado en muchos tipos de aplicaciones. Para proveer una experiencia de usuario familiar y consistente, se deben utilizar las \ac{API}s existentes de Men\'u para exponer las acciones de usuario y otras opciones dentro de las activities.

A partir de la version 3.0 de Android no se requiere un bot\'on de men\'u dedicado como si lo requerian las versiones anteriores.  Con este cambio, las aplicaciones contienen un componente denominado action bar para presentar acciones comunes a los usuarios.

\subsection{Los dialogs}
\label{subsec:dev.dialogs}

El dialog es una peque\~na ventana que toma el primer plano quitandole el foco al activity que le dio origen. Se utiliza para permitir al usuario realizar una toma de decisi\'on o para ingresar informaci\'on adicional. Estas ventanas no utiliza toda la pantalla y es normalmente utilizado para ventos modales que requieren de la decisi\'on de un usuario para poder continuar operando.

El \ac{API} de Android provee dialogs de distintos tipos de acuerdo a la acci\'on que se requiera. Estos pueden ser:

\begin{itemize}
\item AlertDialogs: son dialogs que muestran un t\'itulo, algunos botones y/o una lista de items a seleccionar.
\item ProgressDialogs: dialogs que visualizan el progreso de una carga de datos o el de alg\'un proceso en particular que realice la aplicaci\'on en ese momento.
\item DatePickerDialogs: dialogs con una \ac{UI} predefinida que permiten al usuario seleccionar una fecha determinada.
\item TimePickerDialogs: dialogs para seleccionar un tiempo determinado.
\end{itemize}
 
\subsection{Los resources}
\label{subsec:dev.resources}

Son clases para acceder a los recursos de una aplicaci\'on. Estos recursos son los archivos adicionales y los contenidos est\'aticos que el c\'odigo fuente utiliza, como por ejemplo, los archivos de im\'agenes, las definiciones de layout, los strings utilizados para la \ac{UI}, las instrucciones de animaci\'on, etc\'etera.

A cada resource se le asigna un identificador \'unico las cuales est\'an definidos en la clase \texttt{R} del proyecto (esta clase es autogenerada al momento de la compilaci\'on de la aplicaci\'on). La clase \texttt{R} tiene una clase est\'atica para cada tipo de resource y estos pueden ser accedidos desde el c\'odigo mismo o desde archivos \texttt{xml}.

\subsection{El \texttt{ASyncTask}}
\label{subsec:dev.asynctask}

Las \texttt{ASyncTasks} permiten el uso correcto del thread de \ac{UI}. Estas clases permiten realizar operaciones en background y publicar los resultados luego en el thread de la \ac{UI} sin la necesidad de tener que manejar los threads y tampoco alg\'un tipo de handler. Estas clases suelen ser utilizados idealmente para operaciones peque\~nas (de algunos pocos segundos).

Entre los m\'etodos que contiene un \texttt{ASyncTask} podemos encontrar:

\begin{itemize}
\item \texttt{doInBackground}: se ejecuta en un thread aparte y realiza la operaci\'on en background.
\item \texttt{onPreExecute}: se ejecuta en el thread de la \ac{UI} previo al m\'etodo \texttt{doInBackground}.
\item \texttt{onPostExecute}: se ejecuta en el thread de la \ac{UI} luego de la ejecuci\'on del m\'etodo \texttt{doInBackground}.
\end{itemize}

\subsection{El content provider}
\label{subsec:dev.contentProvider}

Una base de datos SQlite es de car\'acter privada a la aplicaci\'on que la crea. Si uno quisiera compartir datos con otras aplicaciones se puede utilizar un content provider el cual permite compartir los datos de una aplicaci\'on a otras. En muchos casos esta informaci\'on se encuentra almacenada en una base de datos SQlite.

De la misma forma, un content provider puede ser utilizado internamente por la misma aplicaci\'on para acceder a los datos. Para poder tener el acceso a estos datos se debe realizar mediante una \ac{URI}.

Muchos datasources de Android como los contactos de un tel\'efono se acceden mediante content providers.

La base SQlite almacena toda la informaci\'on en un simple archivo. Si uno tiene acceso a este archivo se podr\'ia trabajar directamente con los datos de la base, pero para poder hacerlo se debe solamente utilizar un emulador Android o un dispositivo que se encuentre en el modo "root" el cual permite tener acceso a ciertas carpetas y archivos del sistema Android que un dispositivo est\'andar no lo permite.

\section{Los servicios}
\label{sec:services}

Un servicio es un componente de la aplicaci\'on que puede realizar operaciones long-running o de larga corrida en el background o en segundo plano y no provee de una \ac{UI}.

Otro componente de la aplicaci\'on puede iniciar un servicio y este continuar\'a corriendo en segundo plano a\'un si el usuario cambia a otra aplicaci\'on. Adicionalmente, un componente puede unirse a un servicio para interactuar con este y tambi\'en para realizar \ac{IPC}. Por ejemplo, un servicio podr\'ia manejar transacciones de red, reproducir m\'usica, ejecutar acciones de entrada y salida sobre archivos (file I/O), o interactuar con un content provider, todo desde un segundo plano que resulta invisible al usuario.

Hay que considerar que un servicio corre en el thread principal del proceso que lo contiene. El servicio no crea su propio thread y no corre en procesos separados, a menos que se lo especifique.

\section{Los secretos de performance}
\label{sec:performance.secrets}

Existen una serie de consejos sobre el desarrollo de aplicaciones Android las cuales producen una mejora en la performance de las mismas:

\begin{enumerate}

\item Evitar crear objetos innecesarios: La creaci\'on de objetos siempre tiene un costo. Un garbage collector generacional con pools de asignaci\'on por thread para objectos temporarios puede hacer la asignaci\'on de objetos un poco m\'as eficiente, pero la asignaci\'on de memoria es siempre m\'as costosa que no asignar memoria para los objetos.

\item Es preferible un est\'atico sobre un virtual: Si no es necesario acceder a las variables de un objeto, es recomendable hacer los m\'etodos est\'ticos. Las llamadas tendr\'an un aumento de velocidad entre 15\%-20\%.

\item Usar static final para constantes: Es una buena pr\'actica declarar las constantes como static final cuando sea posible, ya que de esta manera se evita el uso de un m\'etodo de inicializaci\'on de clase. 

\item Evitar getters/setters internos: Llamadas a m\'etodos virtuales son costosas, mucho m\'as que la b\'usqueda de variables instanciadas. Es razonable seguir pr\'acticas de programaci\'on orientadas a objetos y tener getters y setters en una interfaz p\'ublica, pero dentro de una clase uno deber\'ia siempre acceder a las variables directamente.

\item Usar la nueva sintaxis del ciclo \texttt{for}: La nueva sintaxis del ciclo \texttt{for}, conocida tambi\'en como el ciclo \texttt{for-each}, puede ser utilizada para collections que implementen la interfaz \texttt{Iterable} y para arrays. Con las collections, un \texttt{iterator} es asignado para hacer llamadas a la interfaz de \texttt{hasNext} y \texttt{next}. Con un ArrayList, un ciclo con contador es cerca de 3 veces m\'as r\'apido (con o sin \ac{JIT}), pero para otras collections el nuevo ciclo \texttt{for} ser\'a exactamente equivalente al de un \texttt{iterator} expl\'icito.

\item Evitar usar puntos flotantes (floating-points): Los puntos flotantes son 2 veces m\'as lentos que los integers o enteros en los dispositivos Android. En t\'erminos de velocidad, no hay diferencia entre el float y el double en los hardware modernos. En tama\~no, double es 2 veces m\'as grande. Con m\'aquinas de escritorio, asumiendo que el espacio no es un problema, ser\'ia preferible double antes que float. 

\item Saber usar las librerias: Adem\'as de conocer las razones de utilizar una librer\'ia que ya se encuentre disponible antes que armar una propia, hay que tener en cuenta que el sistema ofrece la libertad de poder reemplazar las llamadas a m\'etodos de librer\'ias con otras que contengan c\'odigo assembler, las cuales podr\'ian ser mejores en comparaci\'on con el mejor c\'odigo que el \ac{JIT} pueda producir para su equivalente en Java.

\end{enumerate}

\subsection{Mitos de performance}
\label{subsec:dev.perf.myths}

En dispositivos sin \ac{JIT}, es cierto que invocar m\'etodos a trav\'es de una variable con un tipo exacto que a trav\'es de una interfaz es sutilmente m\'as eficiente (por ejemplo, es mejor invocar m\'etodos en un \texttt{HashMap} que en un \texttt{Map}, a\'un cuando en ambos casos el \texttt{Map} fuera un \texttt{HashMap}). 

En dispositivos sin \ac{JIT}, realizar caching para los accesos a las variables de objetos logra una mejora del 20\% que repetidamente acceder a la misma variable. Con un \ac{JIT}, el acceso a variables cuesta lo mismo que a un acceso a variables de un objeto en forma local, por lo tanto, no es una optimizaci\'on requerida a menos que uno sienta que esto facilita la lectura del c\'odigo (ocurre lo mismo para los tipos \texttt{final}, \texttt{static} y \texttt{static final}).

\subsection{Siempre medir}
\label{subsec:dev.perf.measure.always}

Antes de comenzar a optimizar, hay que asegurarse que uno tenga un problema que necesite resolver. Para esto, es importante medir con detalle la performance existente, porque de otra manera, no se podr\'a medir el beneficio de la implementaci\'on de alguna alternativa de mejora.

Se recomienda utilizar Caliper para correr los microbenchmarks creados por uno mismo. El Caliper es un framework de microbenchmarking para Java.

\section{La interacción con servicios remotos}
\label{sec:remote.services}

El parser de \texttt{JSON} que provee Android es poco performante. Es por eso que para este \ac{TFE} se decidi\'o utilizar una librer\'ia denominada \texttt{GSON} el cual permite tener una mejor performance al parser provisto por Android. Su funci\'on consiste en la de poder convertir objetos Java en representaciones \texttt{JSON} y tambi\'en a su inversa. Esta librer\'ia parte de un proyecto open-source que tuvo en cuenta poder evitar el uso de las annotations de Java ya que no necesita el c\'odigo fuente del objecto a convertir\cite{gson}.

\section{El \ac{DDMS}}
\label{sec:ddms}

Android contiene una herramienta de depuraci\'on llamada \ac{DDMS} que provee servicios de port-forwarding, capturas de pantalla del dispositivo, informaci\'on de threads, memoria, procesos, entre otros. Se encuentra integrado en el entorno Eclipse (se accede mediante una de las perspectivas) y tambi\'en se encuentra en el directorio \texttt{tools} de la \ac{SDK}. Funciona tanto con un emulador como con un dispositivo conectado a la computadora con el entorno Eclipse instalado.

En Android, cada aplicaci\'on corre en su propio proceso, en el que cada uno corre en su propia \ac{MV}. Cada \ac{MV} expone un puerto \'unico en el que un depurador se pueda conectar.

Cuando el \ac{DDMS} inicia, se conecta al \ac{ADB}. Al conectarse un dispositivo, un servicio de monitoreo de la \ac{MV} se crea entre el \ac{ADB} y el \ac{DDMS}, el cual notifica al \ac{DDMS} cuando una \ac{MV} en el dispositivo se inicia o finaliza. Cuando una \ac{MV} est\'a corriendo, el \ac{DDMS} obtiene el identificador de proceso de la \ac{MV}, a trav\'es del \ac{ADB}, y abre una conexi\'on al depurador de la \ac{MV} por medio del \ac{ADBD} en el dispositivo. El \ac{DDMS} ahora puede hablar a la \ac{MV} usando un protocolo customizado.

\section{El \texttt{LogCat}}
\label{sec:logcat}

El sistema de depuraci\'on de Android provee un mecanismo para recolectar y visualizar salidas del sistema depuradas. Los logs de varias aplicaciones y porciones del sistema se recolectan en una serie de buffers circulares, los cuales pueden ser visualizados y filtrados por el \texttt{LogCat} desde una l\'inea de comandos o desde la \ac{UI} del Eclipse.

Entre los m\'etodos que se encuentran para realizar una acci\'on de log, se encuentran:

\begin{itemize}
\item \texttt{Log.e}: se utiliza para registrar errores.
\item \texttt{Log.w}: se utiliza para registrar precauciones o warnings.
\item \texttt{Log.i}: se utiliza para registrar mensajes de informaci\'on.
\item \texttt{Log.d}: se utiliza para registrar mensajes del tipo debug.
\item \texttt{Log.v}: se utiliza para registrar mensajes verbales.
\item \texttt{Log.wtf}: se utiliza para registrar errores muy graves que nunca deberian ocurrir (What a Terrible Failure).
\end{itemize}

La salida b\'asica del \texttt{LogCat} incluye informaci\'on de diferentes fuentes. Para su uso en actividades de depuraci\'on, puede ser muy \'util filtrar la salida del \texttt{LogCat} de acuerdo a diferentes etiquetas o tags para la aplicaci\'on espec\'ifica que se quiera analizar. De esta manera, uno puede enfocarse en los logs de la propia aplicaci\'on y evitar recibir informaci\'on de otras fuentes o partes que para el caso a analizar sean de poco inter\'es.
