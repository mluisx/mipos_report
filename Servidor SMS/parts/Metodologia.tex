% Usar tiempo pasado

\begin{itemize}
	\item \textbf{Desarrollo de una aplicaci\'on Android\Si{\texttrademark} que representa a un \ac{TPV} sobre un \ac{DPT}}\\
	  La primera actividad consisti\'o en el desarroll\'o la \ac{UI} del \ac{TPV}. Luego se codific\'o la l\'ogica de la aplicaci\'on utilizando datos de prueba, o sea, simulando la interacci\'on con el \ac{SMD}. A esta aplicaci\'on se agreg\'o una librer\'ia para poder generar y leer c\'odigos \ac{QR} a trav\'es de la c\'amara \'optica del dispositivo. El uso de la c\'amara tambi\'en se implemento para la toma de fotos de los productos que ingresan en el stock del local y para la toma de im\'agenes de los clientes al ser dados de alta.
	
	\item \textbf{Primeras pruebas de comunicaci\'on entre dispositivos}\\
  	Se realiz\'o una aplicaci\'on de prueba en el \ac{DTC} que solamente conten\'ia desarrollado la comunicaci\'on v\'a Bluetooth\Si{\texttrademark}. Una vez que se pudo comunicar satisfactoriamente esta aplicaci\'on con el \ac{TPV}, se procedi\'o a desarrollar las funciones que permiten el ingreso de datos por parte del cliente que realiza la compra en el local, dando por nombre a esta parte del sistema el de una \ac{TID}.
	
	\item \textbf{Implementaci\'on de un servidor web para manejar peticiones \textbf{\textit{JSON}}}\\
		Para el siguiente paso, se codific\'o una aplicaci\'on \textbf{\textit{Java}} con las librer\'ias adecuadas para poder manejar objetos \textbf{\textit{JSON}} y que esta pueda correr en un servidor de aplicaciones \textbf{\textit{Tomcat}}. Esta aplicaci\'on, conectada a una base de datos \textbf{\textit{MySQL}}, permite almacenar y obtener informaci\'on que luego se env\'ia a trav\'es del servicio web al \ac{TPV}. La base de datos se configur\'o y luego se crearon las tablas y se almacenaron datos para poder utilizarlos en las distintas pruebas de funcionamiento del \ac{SIG}.

	\item \textbf{Inclusi\'on del manejo de transacciones con tarjetas de cr\'editos}\\
		Se agreg\'o la \ac{API} de MercadoPago\Si{\texttrademark} para poder realizar transaciones con tarjetas de cr\'editos y poder tener disponible este m\'etodo de pago en el sistema sin depender de otros gateways de pagos de terceros.

	\item \textbf{Utilizaci\'on de un dispositivo de lectura especial denominado \ac{LTM}}\\
	  Se anex\'o el \ac{LTM} en el \ac{DTC} y se implement\'o una librer\'ia para poder hacer uso del mismo y que pueda realizar la lectura de los datos de las tarjetas de cr\'edito y d\'ebito. Estos datos se env\'ian a el \ac{TPV} para que este pueda ejecutar el pedido de autorizaci\'on correspondiente para la transacci\'on de compra que se quiera llevar a cabo.

	\item \textbf{Incorporaci\'on del uso de la tecnolog\'ia \ac{NFC}}\\
		Para poder utilizar la tecnolog\'ia \ac{NFC} y poder sacar provecho del lector incorporado que trae el \ac{DPT} modelo \textbf{\textit{Nexus}} de marca \textbf{\textit{ASUS}}, se agrego en la aplicaci\'on del \ac{TPV} un m\'odulo que permite hacer lectura y escritura de otros dispositivos que contengan \ac{NFC}. Las pruebas se realizaron mediante unas etiquetas \ac{NFC} que contienen una memoria interna en donde se pudieron almacenar y obtener datos para simular transacciones de compras.
	
	\item \textbf{Proveer al usuario final el an\'alisis de los datos que se manejan diariamente en el \ac{SIG}}\\
		Los gr\'aficos estadi\'isticos se anexaron al \ac{TPV} mediante el uso de una librer\'ia de Google\Si{\texttrademark} que permite, a partir de una serie de datos y unos datos extras, generar im\'agenes que permiten mostrar la evoluci\'on de distintas variables de importancia para el vendedor en un cierto tiempo transcurrido.
	
	\item \textbf{Aprovechar la utilizaci\'on de redes sociales como un canal de marketing}\\
    Por \'ultimo, se incorpor\'o la utilizaci\'on de las \ac{APIs} de Facebook\Si{\texttrademark}, Twitter\Si{\texttrademark} y Foursquare\Si{\texttrademark} en la \ac{TID} para poder brindar al cliente la posibilidad de que pueda ingresar con su cuenta a cualquiera de estas redes sociales y env\'ie un mensaje relacionado con el local a todos sus contactos para que se pueda generar una publicidad viral y que de esta forma pueda servir este \ac{SIG} como una herramienta de marketing.
    	
\end{itemize}