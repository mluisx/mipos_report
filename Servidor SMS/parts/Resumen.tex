% Interlineado simple
% En tiempo presente

% Objetivo General
% Del an�lisis de los resultados surge que ...
% Conclusiones (250 palabras, 1 sola p�gina)


{%\setstretch{0.95}
\slshape

La elaboraci\'on de este \ac{TFE} consiste en el desarrollo de un sistema integral de un \ac{TPV} con administraci\'on de stocks.

Este sistema tiene como objectivo manejar las transacciones de ventas diarias de un local comercial, almacenando datos sobre los productos vendidos, clientes, formas de pagos, ingreso de mercader\'ia, etc\'etera, as\'i como tambi\'en posibilita la visualizaci\'on y manejo de informaci\'on estad\'istico para el control y an\'alisis de los movimientos que surgan en un comercio.

La elaboraci\'on de este \ac{TFE} consiste de tres aplicaciones. Una aplicaci\'on representa al \ac{TPV} implementado en un \acf{DPT} de 7 pulgadas cuyo sistema operativo es Android\Si{\texttrademark}, la otra aplicaci\'on funciona bajo un \ac{DTC} con Android\Si{\texttrademark} y cuya finalidad es la de poder brindar un dispositivo que funcione como \ac{TID} al sistema, y por \'ultimo se desarrolla un \ac{SIG} cuya funci\'on es la de recibir, procesar y enviar informaci\'on desde y hacia el \ac{TPV}. 

Este sistema permite utilizar un dispositivo de \ac{LTM} que funciona conectado a la \ac{TID} como una interfaz de ingreso de datos para el \ac{TPV}. 

La tecnolog\'ia \ac{NFC} se encuentra presente en este \ac{TFE} permitiendo su uso a trav\'es del \ac{TPV} para poder realizar una transacci\'on de compra segura. Se elaboran distintas pruebas con etiquetas \ac{NFC} simulando transacciones de compras cuyos resultados fueron expuestos en este \ac{TFE}.

Se introduce el uso de algunas \ac{APIs} existentes como las de la red social Twitter\Si{\texttrademark} y el sistema de cobro MercadoPago\Si{\texttrademark} para extender las caracter\'isticas que el sistema puede ofrecer a los usuarios.
}
