% Interlineado simple
% En tiempo presente

% Objetivo General
% Del an�lisis de los resultados surge que ...
% Conclusiones (250 palabras, 1 sola p�gina)


{%\setstretch{0.95}
\slshape

La elaboraci\'on de este \ac{TFE} consiste en el desarrollo de un sistema integral de \ac{TPV}.

Este sistema tiene como objectivo manejar las transacciones de ventas diarias de un local comercial, almacenando datos sobre los productos vendidos, clientes, formas de pagos, stock, etc\'etera.

La elaboraci\'on de este \ac{TFE} est\'a compuesto por el desarrollo de tres aplicaciones. La primer aplicaci\'on consiste en el \ac{TPV} implementado en un \acf{DPT} de 7 pulgadas cuyo sistema operativo es Android\Si{\texttrademark}. La segunda aplicaci\'on funciona bajo un \ac{DTC} tambi\'en funcionando con el sistema Android\Si{\texttrademark} y cuya finalidad es la de poder brindar a los clientes un dispositivo donde pueda funcionar como una \ac{TID} al sistema. La \'ultima de las aplicaciones elaboradas consiste en un \ac{SMD} que se encuentra ubicado en un servidor \textbf{\textit{Tomcat}} conectado a una base de datos \textbf{\textit{MySQL}} y cuya funci\'on es la de recibir, procesar y enviar informaci\'on desde y hacia el \ac{TPV}. 

La conexi\'on entre el \ac{TPV} y el servidor se realiza mediante \ac{WS} y la transferencia de los datos a trav\'es de objetos \textbf{\textit{JSON}}, en tanto la comunicaci\'on entre el \ac{TPV} y la \ac{TID} se realiza mediante la tecnolog\'ia Bluetooth\Si{\texttrademark}.

El \ac{TPV} contiene una base de datos local del tipo \textbf{\textit{SQLite}} para poder continuar funcionando en el caso de una falla en la comunicaci\'on con el \ac{SMD}. Al momento de reconectarse se activa un proceso de sincronizaci\'on que transfiere todos los datos almacenados en forma local en el \ac{TPV} al \ac{SMD}.

Este sistema permite utilizar un dispositivo de lectura de tarjetas con bandas magn\'eticas denominado \ac{LTM} que consiste en un dispositivo que se conecta a la entrada de sonido de 3.5 mil\'imetros de la \ac{TID} y actua como un sensor de sonido que permite decodificar las muestras le\'idas y genera los bits necesarios para representar los datos de una tarjeta y poder utilizarlos en el \ac{TPV} y poder procesarlos. 

La tecnolog\'ia \ac{NFC} se encuentra presente en este \ac{TFE} permitiendo su uso a trav\'es del \ac{TPV} para poder realizar una transacci\'on de compra segura. Se elaboraron distintas pruebas con distintas etiquetas \ac{NFC} simulando transacciones de compras y cuyos resultados fueron expuestos en este \ac{TFE}.

Algunas \ac{APIs} como las de Foursquare\Si{\texttrademark}, Facebook\Si{\texttrademark} y Twitter\Si{\texttrademark} se encuentran implementadas para poder utilizarlas como herramientas de marketing para el local comercial que quiera implementar este sistema desarrollado, y como complemento, el \ac{TPV} brinda al usuario un m\'odulo de gr\'aficos estad\'isticos para que sean de utilidad en el caso de hacer un estudio del mercado y la progresi\'on en las actividades desarrolladas en el comercio en cuesti\'on.

}
