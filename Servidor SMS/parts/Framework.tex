%\chapter{Descripci�n del Hardware a Utilizar}
\chapter{Dise�o de framework para el manejo de mensajes \acs{SMS}}
%\sffamily

Debido a la cantidad de mensajes SMS que el servidor debe procesar y para mejorar la modularidad y facilitar la expansi�n del servidor hacia diferentes terminales de trabajo, se pens� en la realizaci�n de un framework que permita separar las tareas de lectura y almacenamiento de los mensajes SMS del procesamiento y env�o de los mismos.

\section{Estructura del framework}
\label{sec:def.Estructura}

La estructura del framework (figura \ref{fig:Estruc.Frw}) se basa en varios threads que permiten el continuo funcionamiento del servidor y la permanente transferencia de los mensajes SMS a los distintos m�dulos para su posterior procesamiento. Su estructura consiste de los siguientes m�dulos:

\figuraoo{Estructura del framework para el manejo de mensajes SMS}{fig:Estruc.Frw}{width=\textwidth}{Framework}{!h}

El m�dulo ``Interfaz De Hardware'' consiste en un thread el cual dado un intervalo de tiempo fijado por el operador de la aplicaci�n, se comunica con el dispositivo GSM y realiza una lectura de los distintos mensajes SMS ``entrantes'' en la memoria del dispositivo donde son almacenados. Luego de realizar la lectura, el m�dulo toma los mensajes le�dos y los env�a a una cola de entrada para su posterior almacenamiento. Esta cola de entrada contiene la implementaci�n de sem�foros para evitar la simultaneidad de acciones que dar�an lugar a inconsistencias en los datos almacenados en la cola.

La memoria definida por defecto en el tel�fono celular utilizado en este TFC es la memoria de estado s�lido interna que contiene (64 KBytes de tama�o). Si el usuario desea, se puede modificar la memoria de almacenamiento de los mensajes SMS (pudiendo ser la Memory Stick o la memoria de la tarjeta SIM) utilizando un comando AT que permite hacerlo. 

Luego de realizar la lectura, el m�dulo verifica si se encuentran mensajes SMS disponibles en una ``cola de salida'' de las mismas caracteristicas a la ``cola de entrada''. Si encuentra uno o mas mensajes, los quita de la cola y los env�a a trav�s del dispositivo GSM a los usuarios finales correspondientes.

El m�dulo ``Consumidor'' esta compuesto de un thread que quita los mensajes almacenados en la cola de entrada en un per�odo fijo de tiempo para luego identificarlos y enviarlos al m�dulo de procesamiento.

El m�dulo ``ProcesoSMS'' contiene un interprete de texto que va tomando las palabras contenidas en los mensajes SMS para despu�s almacenarlos en variables de la aplicaci�n y poder as� realizar las acciones necesarias de acuerdo a estos datos almacenados.

El m�dulo ``Interfaz Grafica 1'' es el encargado de visualizar todas las operaciones que el servidor de SMS va realizando y de los distintos tipos de SMS que va recibiendo y enviando. A su vez, muestra informaci�n relevante a lo que tenga que ver con las reservas de vuelos.

El m�dulo ``Productor'' es el encargado de generar un mensaje SMS de acuerdo a la estructura est�ndar que estos poseen. Una vez generado el mensaje se incorpora el texto dirigido al usuario final del mensaje para despu�s enviarlo a la ``cola de salida'' donde son almacenados. Cuando el m�dulo ``Interfaz De Hardware'' realiza la lectura de los mensajes contenidos por la ``cola de ``salida'', se quitan de la cola y se envian los usuarios finales a travez del dispositivo GSM.

La idea principal de este framework es modularizar todas las acciones referidas a la transferencia de los mensajes SMS en el servidor y permitir la posibilidad de que el servidor pueda ser administrado y visualizado en diferentes terminales de trabajo, pudiendo asi tener la posibilidad de generar distintas interfaces graficas a gusto del operador y permitiendo configurar la distribuci�n de los mensajes SMS en las distintas terminales de acuerdo al tipo de mensaje SMS recibido.