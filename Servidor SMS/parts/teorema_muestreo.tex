\section{El teorema de muestreo}
%\label{sec:teorema.muestreo}
De todos los teoremas y t�cnicas relacionados con la transformada de Fourier, el que ha tenido m�s impacto en la transmisi�n y el procesado de la informaci�n ha sido el teorema de muestreo. Dada una se�al paso bajo $x(t)$ de banda limitada, de forma que no tiene componentes de frecuencia por encima de $\omega_B$\,rad/s, el teorema de muestreo dice que $x(t)$ queda completamente determinada por sus valores en instantes tomados cada $T$ segundos, si se cumple que $T<\pi/\omega_B$. Este teorema nos permite reconstruir completamente una se�al de banda limitada a partir de sus muestras tomadas con una frecuencia de $\omega_S=2\pi/T$, suponiendo que se cumple que $\omega_S$ es mayor a igual que $2\omega_B$, es decir, dos veces la frecuencia m�s alta presente en la se�al de banda limitada $x(t)$. La m�nima frecuencia de muestreo $2\omega_B$ se denomina frecuencia de Nyquist.~\cite{soliman.senales.sistemas}
