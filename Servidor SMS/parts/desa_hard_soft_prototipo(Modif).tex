\chapter{Dise�o hardware y configuraci�n software del prototipo de la \acs{IH}}
\label{sec:proto}
Esta parte se centra en el dise�o del hardware de la \ac{IH}. Esto involucra la configuraci�n de los recursos de hardware que se encuentran incorporados en el \ac{uC} tanto as� como el dise�o esquem�tico de los componentes discretos externos.

Desde un punto de vista global se observa en la figura \ref{fig:Host-IH}
\figuraoo{V�nculo entre el \acs{HOST} y la IH}{fig:Host-IH}{width=\textwidth}{Gsm_network}{!h}
la relaci�n que existe entre el \acfn{HOST} y la \ac{IH}. Este v�nculo es un puerto \ac{UART} a una velocidad de 115200\un{bps}. Es \textsl{full-duplex} y utiliza 2 l�neas de datos (Rx y Tx) y una l�nea de referencia de masa. Los niveles que se manejan son \acsu{TTL} y no se hace uso de los niveles RS-232C debido a que la comunicaci�n es entre dos placas muy pr�ximas. En la figura \ref{fig:TTL-RS232C}
\figuraot{Relaci�n de tensiones \acsu{TTL} y RS-232C}{fig:TTL-RS232C}{width=0.7\textwidth}{TTL-RS232C}{!t}
se observa la correspondencia entre las se�ales TTL y RS-232C.

\section{Diagrama en bloques de la \acs{IH}}
\label{sec:proto.digrama}

A grandes rasgos se realiza un diagrama que tiene como n�cleo al \ac{uC}. A �ste se conectan los diferentes recursos de la \ac{IH} a controlar. En la figura \ref{fig:diag-block-IH}
\figuraot{Diagrama en bloques de la \ac{IH}}{fig:diag-block-IH}{width=\textwidth}{diag-block-IH}{!t}
se observa dicho diagrama en el cual figuran conexiones  entre cada recurso y el \ac{uC}. Un recurso se refiere a un perif�rico como ser el teclado, el \ac{LCD}, etc. Dichas l�neas son en su mayor�a digitales siendo s�lo una de entrada anal�gica.


\subsection{Descripci�n del puerto \acs{UART}}
\label{sec:des.UART}
En la figura \ref{fig:diag-block-IH} se observan dos l�neas etiquetadas como \ac{UART}. Estas l�neas transmiten bits a una tasa ideal de $V_{I}=115200\un{bps}$ totales, de los cuales s�lo 8\un{bits} de 10\un{bits} son de datos �tiles. Esto nos da una tasa ideal de datos �tiles $V_{IU}=115200\un{bps}*8\un{bits}/10\un{bits}=92160\un{bps}$ como m�ximo, debido a que la configuraci�n elegida para este puerto serie es 8N1:
\begin{itemize}
	\item 1\un{bit} de inicio
	\item 8\un{bits} de datos
	\item Sin bit de paridad
	\item 1\un{bit} de parada
\end{itemize}

